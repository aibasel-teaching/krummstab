\usetikzlibrary{lindenmayersystems}
\tikzset{%
    stab line/.style={cap=round, line join=round,},
    stab/.pic={
        \draw[rotate=90] lindenmayer system[lindenmayer
            system={krummstab, step=.1pt, angle=4, axiom=F, order=18}];
        \draw[overlay, line width=55pt] (0, 0) -- (0, -1);}
}%
\pgfdeclarelindenmayersystem{krummstab}{%
    % The fact that these \draw commands actually affect the styling is a
    % hack I came across accidentally. I'm not sure how and why it works.
    \symbol{F}{\draw[stab line, line width=55pt];\pgflsystemdrawforward}
    \symbol{A}{\draw[stab line, line width=34pt];\pgflsystemdrawforward}
    \symbol{R}{\draw[stab line, line width=21pt];\pgflsystemdrawforward}
    \symbol{L}{\pgflsystemdrawforward}
    \symbol{B}{\pgflsystemdrawforward}
    \symbol{S}{\pgflsystemdrawforward}
    \rule{F -> LLLL[+MF][-NA]}
    \rule{M -> +M}
    \rule{O -> ++O}
    \rule{N -> -N}
    \rule{L -> LX}
    \rule{X -> L}
    \rule{A -> BB[-NA][+MR]}
    \rule{B -> BY}
    \rule{Y -> B}
    \rule{R -> S[+MR]}
    \rule{S -> ST}
    \rule{T -> S}
}%
